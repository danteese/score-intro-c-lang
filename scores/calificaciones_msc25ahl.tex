\documentclass[11pt]{article}
\usepackage[utf8]{inputenc}
\usepackage[T1]{fontenc}
\usepackage[spanish]{babel}
\usepackage{lmodern}
\usepackage{textcomp}
\usepackage{selinput}
\SelectInputMappings{
  adieresis={ä},
  eacute={é},
  ntilde={ñ},
  uacute={ú},
  iacute={í},
  oacute={ó},
  aacute={á},
  egrave={è},
  igrave={ì},
  ograve={ò},
  agrave={à},
  ccedilla={ç},
  uumlaut={ü},
  oumlaut={ö},
  aumlaut={ä}
}
\usepackage[letterpaper,top=3cm,bottom=3cm,left=2cm,right=2cm]{geometry}
\usepackage{xcolor}
\usepackage{graphicx}
\usepackage{fancyhdr}
\usepackage{listings}
\usepackage{booktabs}
\usepackage{array}
\usepackage{setspace}

% Sin indentación en párrafos
\setlength{\parindent}{0pt}
\setlength{\parskip}{0.5em}

% Control directo del espacio inferior - REMOVED (conflicts with geometry)

% Sin indentación en títulos
\usepackage{titlesec}
\titleformat{\section}{\Large\bfseries}{}{0pt}{}
\titleformat{\subsection}{\large\bfseries}{}{0pt}{}
\titleformat{\subsubsection}{\normalsize\bfseries}{}{0pt}{}

% Sin indentación en listas
\usepackage{enumitem}
\setlist{leftmargin=0pt,itemindent=0pt}

% Sin indentación en todo el documento
\raggedright

% Sin indentación en listings
\lstset{
    basicstyle=\small,
    breaklines=true,
    frame=none,
    backgroundcolor=\color{white},
    commentstyle=\color{commentgreen},
    keywordstyle=\color{codeblue},
    stringstyle=\color{red},
    showstringspaces=false,
    numbers=none,
    tabsize=2,
    columns=flexible,
    keepspaces=true,
    fontadjust=true,
    xleftmargin=0pt,
    xrightmargin=0pt
}

% Colores personalizados
\definecolor{codeblue}{RGB}{41, 128, 185}
\definecolor{commentgreen}{RGB}{39, 174, 96}
\definecolor{scoreorange}{RGB}{230, 126, 34}
\definecolor{headerblue}{RGB}{52, 73, 94}
\definecolor{lightgray}{RGB}{245, 245, 245}


% Headers y footers
\pagestyle{fancy}
\fancyhf{}
\fancyhead[L]{\includegraphics[height=1cm]{../public/ibero.png}}
\fancyhead[R]{\textbf{Reporte de Calificaciones}}
\fancyfoot[C]{\thepage}
\renewcommand{\headrulewidth}{0pt}
\renewcommand{\footrulewidth}{0pt}
\setlength{\headheight}{35pt}

% Footer positioning - geometry package handles this
% \setlength{\footskip}{1cm}

\begin{document}

% Título principal
\begin{center}
\Large\textbf{\color{headerblue}Reporte de Calificaciones}\\[0.5cm]
\large\textbf{MSC25AHL}\\[0.3cm]
\normalsize Fecha de evaluación: 15 de September de 2025
\end{center}

\vspace{0.5cm}
\hrule
\vspace{0.5cm}


\section*{\textbf{1.} Operaciones.c}

\begin{minipage}{\textwidth}
\textbf{Calificación:} \textcolor{commentgreen}{\textbf{8/10}}\\[0.3cm]

\textbf{Comentarios del evaluador:}\\[0.2cm]
\begin{minipage}{\textwidth}
\small
El código ha sido implementado utilizando funciones para realizar cada operación correctamente, lo cual refleja un buen razonamiento lógico. Sin embargo, sería ideal que incluyeras algunos comentarios descriptivos para mejorar la legibilidad del código. Además, considera usar fscanf y fprintf explícitamente para la entrada y salida, ya que esto es una buena práctica que facilitará trabajar con archivos en el futuro. Por último, asegúrate de verificar la división por cero en la función 'division' para evitar comportamientos indeseados.
\end{minipage}
\end{minipage}

\vspace{0.5cm}
\hrule
\vspace{0.5cm}


\section*{\textbf{2.} Resistencia.c}

\begin{minipage}{\textwidth}
\textbf{Calificación:} \textcolor{red}{\textbf{0/10}}\\[0.3cm]

\textbf{Comentarios del evaluador:}\\[0.2cm]
\begin{minipage}{\textwidth}
\small
No se ha encontrado el archivo resistencia.c. Por favor, asegúrate de entregar todos los archivos requeridos para que pueda ser evaluado. Recuerda que cada ejercicio cuenta para tu calificación final.
\end{minipage}
\end{minipage}

\vspace{0.5cm}
\hrule
\vspace{0.5cm}


\section*{\textbf{3.} Conversioncmsmts.c}

\begin{minipage}{\textwidth}
\textbf{Calificación:} \textcolor{commentgreen}{\textbf{9/10}}\\[0.3cm]

\textbf{Comentarios del evaluador:}\\[0.2cm]
\begin{minipage}{\textwidth}
\small
El código está bien estructurado y funcional, utilizando funciones para manejar las conversiones adecuadamente. Sin embargo, una mejora sería definir las funciones al inicio y agregar comentarios para explicar cada parte del código. Además, considera usar constantes descriptivas y un estilo de input/output consistente; esto te ayudará a encaminarte hacia el uso profesional del lenguaje C. En general, ha hecho un buen trabajo y está en el camino correcto.
\end{minipage}
\end{minipage}

\vspace{0.5cm}
\hrule
\vspace{0.5cm}


\section*{\textbf{4.} Conversionseghms.c}

\begin{minipage}{\textwidth}
\textbf{Calificación:} \textcolor{scoreorange}{\textbf{7/10}}\\[0.3cm]

\textbf{Comentarios del evaluador:}\\[0.2cm]
\begin{minipage}{\textwidth}
\small
El código funciona y cumple con el requisito de convertir segundos a formato HH:MM:SS. Sin embargo, las funciones para calcular horas y minutos están usando la misma lógica, lo que puede llevar a errores. Deberías revisar la lógica de la función de minutos para que devuelva el valor correcto. Sería beneficioso estructurar el código utilizando comentarios y asegurándote de definir las funciones al principio. Buen esfuerzo, sigue practicando para mejorar.
\end{minipage}
\end{minipage}

\vspace{0.5cm}
\hrule
\vspace{0.5cm}


\section*{\textbf{Resumen General}}

\begin{center}
\begin{tabular}{l r}
\toprule
\textbf{Métrica} & \textbf{Valor} \\\\
\midrule
Calificación Total & 24/40 \\\\
Promedio & 6.00/10 \\\\
\bottomrule
\end{tabular}
\end{center}

\vspace{1cm}
\begin{center}
\textbf{Prof. Edgar Ortiz}\\[0.2cm]
\small\textit{Sistema de Evaluación Automática}
\end{center}

\vfill
\vspace{3cm}

\end{document}
